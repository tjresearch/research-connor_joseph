\documentclass{article}
\usepackage[utf8]{inputenc}

\title{Journal 0}
\author{Connor Grimberg}
\date{August 2019}

\begin{document}

\maketitle


Option B:

1. My proposed idea was to determine based on the run history of a user, if a program they just ran was actually them or some kind of malware. I planned on determining the irregularities by how far off they were from the prediction of an autoregressive moving average model which predicts the future value of a series. I planned on testing this using a few real networks and my own virtual network that I would set up.

2. I want to change my idea because I feel like if it works, it will be too simple because all I do is plug numbers into a formula. There is also the chance that the way I am solving the problem can't be reliable enough to pass any tests I give it. Testing this is also going to be difficult because I want to set up my own network and run normal and abnormal programs on the virtual machines. Figuring out how to automatically do that with variation seems very difficult. 

3. I don't have any good new ideas. I could give my project the ability to look for cases of weird behavior in other parts of the computer and not just the run history. 

4. Hackers do several things to breach a computer's security including running programs to facilitate the breach, and hiding in these programs can allow them to be ignored by standard antivirus programs. A technology being worked on right now is behavior-based antivirus that checks multiple things such as, which programs are run and at what time, if a user suddenly asks for privilege escalation, and other unusual behavior. My project is to determine the normal behavior of a user and the network based on what and when programs are run and use a formula that calculates expected values over time to return warnings about suspicious behavior.


\end{document}
